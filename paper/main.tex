\documentclass[a4paper,14pt]{article}

%%% Работа с русским языком
\usepackage{cmap}					% поиск в PDF
\usepackage{mathtext} 				% русские буквы в формулах
\usepackage[T2A]{fontenc}			% кодировка
\usepackage[utf8]{inputenc}			% кодировка исходного текста
\usepackage{algorithm}
\usepackage{algorithmic}
\usepackage{caption}
\usepackage[english,russian]{babel}	% локализация и переносы
\usepackage{indentfirst}
\frenchspacing

\newcommand{\bz}{\mathbf{z}}
\newcommand{\bx}{\mathbf{x}}
\newcommand{\by}{\mathbf{y}}
\newcommand{\bv}{\mathbf{v}}
\newcommand{\bw}{\mathbf{w}}
\newcommand{\ba}{\mathbf{a}}
\newcommand{\bb}{\mathbf{b}}
\newcommand{\bp}{\mathbf{p}}
\newcommand{\bq}{\mathbf{q}}
\newcommand{\bt}{\mathbf{t}}
\newcommand{\bu}{\mathbf{u}}
\newcommand{\bT}{\mathbf{T}}
\newcommand{\bX}{\mathbf{X}}
\newcommand{\bZ}{\mathbf{Z}}
\newcommand{\bS}{\mathbf{S}}
\newcommand{\bH}{\mathbf{H}}
\newcommand{\bW}{\mathbf{W}}
\newcommand{\bY}{\mathbf{Y}}
\newcommand{\bU}{\mathbf{U}}
\newcommand{\bQ}{\mathbf{Q}}
\newcommand{\bP}{\mathbf{P}}
\newcommand{\bA}{\mathbf{A}}
\newcommand{\bB}{\mathbf{B}}
\newcommand{\bC}{\mathbf{C}}
\newcommand{\bE}{\mathbf{E}}
\newcommand{\bF}{\mathbf{F}}
\newcommand{\bomega}{\boldsymbol{\omega}}
\newcommand{\btheta}{\boldsymbol{\theta}}
\newcommand{\bgamma}{\boldsymbol{\gamma}}
\newcommand{\bdelta}{\boldsymbol{\delta}}
\newcommand{\bPsi}{\boldsymbol{\Psi}}
\newcommand{\bpsi}{\boldsymbol{\psi}}
\newcommand{\bxi}{\boldsymbol{\xi}}
\newcommand{\bchi}{\boldsymbol{\chi}}
\newcommand{\bzeta}{\boldsymbol{\zeta}}
\newcommand{\blambda}{\boldsymbol{\lambda}}
\newcommand{\beps}{\boldsymbol{\varepsilon}}
\newcommand{\bZeta}{\boldsymbol{Z}}
% mathcal
\newcommand{\cX}{\mathcal{X}}
\newcommand{\cY}{\mathcal{Y}}
\newcommand{\cW}{\mathcal{W}}

\newcommand{\dH}{\mathbb{H}}
\newcommand{\dR}{\mathbb{R}}
\newcommand{\dE}{\mathbb{E}}
% transpose
\newcommand{\T}{^{\mathsf{T}}}

\renewcommand{\epsilon}{\ensuremath{\varepsilon}}
\renewcommand{\phi}{\ensuremath{\varphi}}
\renewcommand{\kappa}{\ensuremath{\varkappa}}
\renewcommand{\le}{\ensuremath{\leqslant}}
\renewcommand{\leq}{\ensuremath{\leqslant}}
\renewcommand{\ge}{\ensuremath{\geqslant}}
\renewcommand{\geq}{\ensuremath{\geqslant}}
\renewcommand{\emptyset}{\varnothing}

%%% Дополнительная работа с математикой
\usepackage{amsmath,amsfonts,amssymb,amsthm,mathtools} % AMS
\usepackage{icomma} % "Умная" запятая: $0,2$ --- число, $0, 2$ --- перечисление

%% Номера формул
%\mathtoolsset{showonlyrefs=true} % Показывать номера только у тех формул, на которые есть \eqref{} в тексте.
%\usepackage{leqno} % Нумереация формул слева

%% Свои команды
\DeclareMathOperator{\sgn}{\mathop{sgn}}

%% Перенос знаков в формулах (по Львовскому)
\newcommand*{\hm}[1]{#1\nobreak\discretionary{}
	{\hbox{$\mathsurround=0pt #1$}}{}}

%%% Работа с картинками
\usepackage{graphicx}  % Для вставки рисунков
\setlength\fboxsep{3pt} % Отступ рамки \fbox{} от рисунка
\setlength\fboxrule{1pt} % Толщина линий рамки \fbox{}
\usepackage{wrapfig} % Обтекание рисунков текстом

%%% Работа с таблицами
\usepackage{array,tabularx,tabulary,booktabs} % Дополнительная работа с таблицами
\usepackage{longtable}  % Длинные таблицы
\usepackage{multirow} % Слияние строк в таблице

%%% Теоремы
\theoremstyle{plain} % Это стиль по умолчанию, его можно не переопределять.
\newtheorem{theorem}{Теорема}[section]
\newtheorem{proposition}[theorem]{Утверждение}

\theoremstyle{definition} % "Определение"
\newtheorem{corollary}{Следствие}[theorem]
\newtheorem{problem}{Задача}[section]

\theoremstyle{remark} % "Примечание"
\newtheorem*{nonum}{Решение}

%%% Программирование
\usepackage{etoolbox} % логические операторы

%%% Страница
\usepackage{extsizes} % Возможность сделать 14-й шрифт
\usepackage{geometry} % Простой способ задавать поля
\geometry{top=25mm}
\geometry{bottom=35mm}
\geometry{left=35mm}
\geometry{right=20mm}
%
%\usepackage{fancyhdr} % Колонтитулы
% 	\pagestyle{fancy}
%\renewcommand{\headrulewidth}{0pt}  % Толщина линейки, отчеркивающей верхний колонтитул
% 	\lfoot{Нижний левый}
% 	\rfoot{Нижний правый}
% 	\rhead{Верхний правый}
% 	\chead{Верхний в центре}
% 	\lhead{Верхний левый}
%	\cfoot{Нижний в центре} % По умолчанию здесь номер страницы

\usepackage{setspace} % Интерлиньяж
%\onehalfspacing % Интерлиньяж 1.5
%\doublespacing % Интерлиньяж 2
%\singlespacing % Интерлиньяж 1

\usepackage{lastpage} % Узнать, сколько всего страниц в документе.

\usepackage{soul} % Модификаторы начертания

\usepackage{hyperref}
\usepackage[usenames,dvipsnames,svgnames,table,rgb]{xcolor}
\hypersetup{				% Гиперссылки
	unicode=true,           % русские буквы в раздела PDF
	pdftitle={Заголовок},   % Заголовок
	pdfauthor={Автор},      % Автор
	pdfsubject={Тема},      % Тема
	pdfcreator={Создатель}, % Создатель
	pdfproducer={Производитель}, % Производитель
	pdfkeywords={keyword1} {key2} {key3}, % Ключевые слова
	colorlinks=true,       	% false: ссылки в рамках; true: цветные ссылки
	linkcolor=red,          % внутренние ссылки
	citecolor=black,        % на библиографию
	filecolor=magenta,      % на файлы
	urlcolor=cyan           % на URL
}

\usepackage{csquotes} % Еще инструменты для ссылок

% \usepackage[style=authoryear,maxcitenames=2,backend=biber,sorting=nty]{biblatex}
% \addbibresource{references.bib}
\usepackage[numbers]{natbib}

\usepackage{multicol} % Несколько колонок

\usepackage{tikz} % Работа с графикой
\usepackage{pgfplots}
\usepackage{pgfplotstable}


\author{Ivan Ilyin, Kirill Semkin, Alexander Terentyev, Vadim Strijov}
\title{\textbf{Исследование нестационарных и неоднородных динамических систем}}
\date{\today}

\begin{document}
	\maketitle
	\section{Abstract}
	В работе исследуется задача восстановления скрытой динамической системы по временному ряду. Предполагается, что динамическая система параметризована. Для нахождения параметра системы используется метод Neural ODE. Строятся фазовые траектории в зависимости от начальных условий и параметра, восстанавливается параметр в случае зависимости его от времени. Проводится анализ точек разладки временного ряда по полученному параметру системы. Эксперимент проведён на синтетических данных и датасетах Run or Walk, The Weather Dataset.

	$\mathbf{Keywords:}$ Machine Learning, Time Series, Neural ODE.
        
	\section{Introduction}
        Одним из способов исследования временных рядов является анализ скрытых состояний динамической системы, порождающей этот ряд. Временной ряд ~--- последовательно измеренные через некоторые (зачастую равные) промежутки времени данные. Динамическая система ~--- это математическая модель, заданная дифференциальным уравнением 
        \begin{equation}
            \frac{d}{dt}\mathbf{X}(t) = f(\mathbf{X}(t), t, \mathbf{w})
        \end{equation}
        Если правая часть уравнения явно зависит от $t$, то динамическую систему называют нестационарной. А если правая часть уравнения представляется в виде
        \begin{equation}
        f(\mathbf{X}, t, \mathbf{w}) = A(t)\mathbf{X}(t) + B(t, \mathbf{w}), \quad B(t, \mathbf{w}) \not\equiv 0
        \end{equation} то динамическую систему называют неоднородной.
        
        На текущий момент существует несколько подходов к восстановлению состояния системы.

        Одним из методов является использование RNN \citep{strobelt2017lstmvis}, однако основным недостатком подхода является дискретное представление изменения состояния скрытой системы, а также неустойчивость к нерегулярно наблюдаемым данным \citep{rubanova2019latentodesirregularlysampledtime}. Также скрытые состояния RNN трудно интерпретируемы \citep{ming2017understanding} \citep{garcia2021visual}.

        В \citep{NEURIPS2018_69386f6b} вводятся нейронные обыкновенные дифференциальные уравнения, которые представляют скрытые состояния в виде динамической системы, данный подход позволяет исследовать нерегулярно наблюдаемые данные \citep{NEURIPS2019_42a6845a}, а также восстановить параметр динамической системы, однако не исследуются динамические системы с параметром, меняющимся во времени.

        В теории оптимальных процессов \citep{pontryagin} вводится управляемый процесс
        \begin{equation}
            \frac{d}{dt}\mathbf{X}(t) = f(\mathbf{X}(t), \mathbf{u}(t)), \quad \mathbf{X}(t_0) = x_0
        \end{equation}
        и задача минимизации функционала
         \begin{equation}
            J = \int_{t_0}^{t_1} L(\mathbf{X}(t), \mathbf{u}(t)) \, dt,
        \end{equation}
        где \( \mathbf{X}(t) \in \mathbb{R}^n \) — вектор состояния системы в момент времени \( t \), \( \mathbf{u}(t) \in \mathbb{R}^r \) — вектор управления в момент времени \( t \), \( f(\mathbf{X}(t), \mathbf{u}(t)) \) — функция, задающая динамику системы. Однако не рассматривается случай, когда $\mathbf{X}(t)$ имеет шум.

        В работе предлагается метод восстановления параметра динамической системы, меняющегося во времени, путем параметризации его производной. Восстанавливается временной ряд по параметру производной динамической системы, исследуются точки разладки с помощью полученного изменения параметра системы во времени.
        

        \section{Problem statement}
        Пусть задан \( \mathcal{D} = \left( \mathbf{\widetilde{X}}_t \mid t \in \{ t_i \}_{i=1}^N \right) \) — временной ряд. 
        
        \[
        \widetilde{\mathbf{X}}_t = \mathbf{X}_t + \varepsilon_t, \quad 
        \varepsilon_t \overset{\text{i.i.d.}}{\sim} \mathcal{N}(0, 1),
        \]
        
        где \( \mathbf{X}_t \) порожден динамической системой (1)
        
        \begin{equation}
        \frac{d}{dt}
        \begin{pmatrix}
        \mathbf{X}(t) \\
        \mathbf{w}(t)
        \end{pmatrix}
        =
        \begin{pmatrix}
        f(\mathbf{X}(t), \mathbf{w}(t)) \\
        v_\theta(t)
        \end{pmatrix},
        \end{equation}
        
        где \( v_\theta(t) \) — параметризованная динамика.
        
        Пусть задан лосс
        
        \begin{equation}
        \mathcal{L}(\mathbf{X}, \widehat{\mathbf{X}}) = \frac{1}{N} \sum_{i=1}^{N} \| \mathbf{X}_i - \widehat{\mathbf{X}}_i \|_2^2
        \end{equation}
        
        Необходимо найти параметр \( \widehat{\theta} \), такой что
        
        \begin{equation}
        \widehat{\theta} = \arg\min_{\theta} \mathcal{L}(\mathbf{X}, \widehat{\mathbf{X}})
        \end{equation}

        \section{Method}

        \begin{algorithm}[H]
        \caption{Обучение параметра \( \theta \)}
        \begin{algorithmic}[1]
        \STATE \textbf{Вход:} временной ряд \( \mathcal{D} \), динамическая система \( f(\mathbf{X}, \mathbf{w}) \), производная скрытого состояния \( v_\theta(t)\), начальные условия \( \mathbf{X_0}, \mathbf{w_0} \)
        \STATE \textbf{Выход:} \( \widehat{\theta} \)
        
        \STATE Инициализировать \( h(t) = (f(\mathbf{X}(t), \mathbf{w}(t)), v_\theta(t)) \)
        \STATE Инициализировать \(\widehat{\theta}\)
        \FOR{epoch = 1 до \texttt{max\_epochs}}
            \FOR{batch в \texttt{batches}}
                \STATE \( (\widehat{\mathbf{X}}, \widehat{\mathbf{w}} ) \gets \mathrm{NeuralODE}(h, (\mathbf{X_0}, \mathbf{w_0}), \{ t_i \}_{i=1}^N) \)
                \STATE Вычислить ошибку \( \mathcal{L}(\mathbf{X}, \widehat{\mathbf{X}}) \)
                \STATE Обновить $\widehat{\theta}$
            \ENDFOR
        \ENDFOR
        \RETURN $\widehat\theta$
        
        \end{algorithmic}
        \end{algorithm}

        \section{Computational experiment}

        Рассмотрим модель математического маятника с изменяющимся ускорением свободного падения и постоянной длиной, заданную дифференциальным уравнением
        
        \begin{equation}
            \frac{d}{dt}
            \begin{pmatrix}
            \varphi(t) \\
            \omega(t) \\
            g(t)
            \end{pmatrix}
            =
            \begin{pmatrix}
            \omega(t) \\
             - \frac{g(t)}{l}sin(\varphi(t)) \\
            3at^2 + 2bt + c
            \end{pmatrix}
        \end{equation}

        С начальными условиями

        \begin{equation}
            \begin{pmatrix}
            \varphi(0) \\
            \omega(0) \\
            g(0)
            \end{pmatrix}
            =
            \begin{pmatrix}
            \frac{\pi}{4} \\
            1 \\
            9,81
            \end{pmatrix}
        \end{equation}

        где $\varphi(t)$ ~--- угол отклонения маятника в момент $t$, $\omega(t)$ ~--- угловая скорость маятника в момент $t$,
        $l$ ~--- постоянная длина маятника в момент, $g(t)$ ~--- меняющееся во времени ускорение свободного падения, $\theta = \begin{pmatrix} a \\ b \\ c\end{pmatrix}$ - обучаемый параметр динамической системы.

        Для проведения эксперимента сгенерируем 100 векторов $\begin{pmatrix} \varphi(t) \\ \omega(t) \end{pmatrix}$ при $\begin{pmatrix} a \\ b \\ c\end{pmatrix} = \begin{pmatrix} -0.01 \\ 0.1 \\ 0.5 \end{pmatrix}$ равномерно на временном интервале $t \in [0, 10]$, решая дифференциальное уравнение (3)-(4) методом Рунге-Кутты четвертого порядка.

        При помощи Neural ODE обучим систему (3)-(4) на сгенерированных данных и проанализируем $\theta$, изменение $g(t)$ во времени, а также лосс полученной системы.

        
        
        
        
	\nocite{*}
        \bibliographystyle{unsrt} % Use a numbered bibliography style
        \bibliography{references.bib}
\end{document}